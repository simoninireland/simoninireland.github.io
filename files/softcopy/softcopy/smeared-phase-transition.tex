\documentclass{beamer}

\usepackage{amsmath}
\usepackage{hyperref}
\usepackage{bibentry}
\usepackage{graphicx}
\usepackage{nameref}
\usepackage{wasysym}
%\usepackage{pbox}
%\usepackage{framed}
\usepackage{listings}
\usepackage{xcolor}

\newtheorem{defn}{Definition}


% ---------- Title etc ----------

\title{A possible smeared phase transition in epidemic track-and-trace}
\author{Simon Dobson}
\institute{%
  \center
  \parbox[t]{5cm}{%
    \center
    School of Computer Science \\ University of St Andrews \\ Scotland UK \\[0.25cm]
    \url{simon.dobson@st-andrews.ac.uk} \\[0.25cm]
    \includegraphics[width=5cm,keepaspectratio]{03-foundation-colour}
  }
}
\date{}


% ---------- Theming ----------

\makeatletter

\usetheme{bjeldbak}
%\setbeamercovered{invisible}

% on-going tables of contents per section
\AtBeginSection[]{%
  \begin{frame}{Structure of this talk}
    \tableofcontents[currentsection] 
  \end{frame}
}

% citations as footnotes and entered into bibliography 
\renewcommand{\cite}[1]{
  %\renewcommand{\thefootnote}{}
  \footnote[frame]{\tiny{\bibentry{#1}}}
}
\setbeamertemplate{bibliography entry title}{}
\setbeamertemplate{bibliography entry location}{}
\setbeamertemplate{bibliography entry note}{}
\nobibliography*

% colour theme
\definecolor{cornsilk}{RGB}{255,248,220}
\definecolor{floral-white}{RGB}{255,250,240}
\setbeamercolor{background canvas}{bg=floral-white}

% show hyperlinks
\hypersetup{%
  colorlinks=true,
  linkcolor=blue,
  urlcolor=blue
}

% code style 
\definecolor{codegreen}{rgb}{0,0.6,0}
\lstset{%
  language=Python,
  basicstyle=\ttfamily\tiny,
  keywordstyle=\color{codegreen},
  commentstyle=\color{magenta},
  breaklines=false,
  keepspaces=true,
}

%\renewcommand{\emph}[1]{\textbf{#1}}

\makeatother
  

% ---------- The slides ----------

\begin{document}


% ---------- Frontmatter ----------

\begin{frame}
  \titlepage
\end{frame}

\begin{frame}
  \frametitle{Introduction}

  \begin{columns}
    \begin{column}{.88\textwidth}
      \begin{block}{Exploring epidemics using network science\cite{em-book} }
        \begin{itemize}
        \item Disease models and parameters
        \item Contact structures
        \item Countermeasures
        \end{itemize}
      \end{block}

      \begin{block}{In the course of writing about these ideas I stumbled
          across something I can't readily explain}
        \begin{itemize}
        \item A change on behaviour that should be crisp, but doesn't
          seem to be
        \item Deserves more exploration
        \end{itemize}
      \end{block}      
    \end{column}
    \begin{column}[t]{.12\textwidth}
      \includegraphics[width=\textwidth,keepaspectratio]{front}
    \end{column}
  \end{columns}

\end{frame}


% ---------- Background ----------

\section{Background}

\subsection{Measuring diseases}

\begin{frame}
  \frametitle{Real diseases -- 1}

  \begin{center}
    \includegraphics[width=.6\textwidth,keepaspectratio]{disease-periods}          
  \end{center}

  \begin{block}{Different periods}
    \begin{itemize}
    \item \emph{Incubation}: from infection to onset of symptoms
    \item \emph{Latent}: from exposure to infectiousness
    \item \emph{Infectious}: overlapping with symptoms (usually)
    \end{itemize}
  \end{block}
  
  \begin{block}{Periods defined by biology, of both disease and host}
  \end{block}
\end{frame}

\begin{frame}
  \frametitle{Real diseases -- 2}

  \begin{center}
    \includegraphics[width=\textwidth,keepaspectratio]{disease-types}          
  \end{center}

\end{frame}

\begin{frame}
  \frametitle{$\mathcal{R}$ and all that\cite{R}}

  \begin{block}{$\mathcal{R}$, the case reproduction number}
    \begin{itemize}
    \item Number of secondary cases per primary
    \item Especially $\mathcal{R}_0$, reproduction absent countermeasures
    \end{itemize}
  \end{block}

  \begin{block}{$r$, the case reproduction rate}
    \begin{itemize}
    \item Doubling time for an epidemic
    \item Also sometimes see $T_g$, the inter-generation time
    \end{itemize}
  \end{block}

  \begin{block}{$k$, the overdispersion parameter}
    \begin{itemize}
    \item Essentially a variance of the distribution around $R$
    \item Small number of individuals can drive most cases
    \end{itemize}
  \end{block}
\end{frame}

\begin{frame}
  \frametitle{The ``wickedness'' of \textsc{covid-19}}

  \begin{block}{$\mathcal{R}_0 \approx 3$ is not particularly infectious}
    \begin{itemize}
    \item Straightforward to get $\mathcal{R} \approx 1.5$; harder to get $\mathcal{R} < 1$
    \item Significant overdispersion
    \item Infection seems to convey only temporary immunity
    \end{itemize}
  \end{block}

  \begin{block}{Case fatality rate is about 1\%}
    \begin{itemize}
    \item Too large to comfortably ignore, but not so large as to
      admit no arguments
    \end{itemize}
  \end{block}

  \begin{block}{Substantial asymptomatic transmission} 
    \begin{itemize}
    \item Asymmetric costs (spreading \textit{vs} dying, ``long \textsc{covid}'')
    \item Effective countermeasures are collective (and expensive)
    \end{itemize}
  \end{block}
\end{frame}


\subsection{Compartmented models of disease}

\begin{frame}
  \frametitle{The goals of modelling}

  \begin{block}{What are we trying to find out?}
    \begin{itemize}
    \item Concrete: how will this \textit{particular} outbreak behave, in this
      \textit{particular} population?
    \item Abstract: how can diseases behave \textit{in general}? Are
      there common mathematical structures?
    \end{itemize}
  \end{block}
\end{frame}

\begin{frame}
  \frametitle{Compartmented models}

  \begin{block}{Traditional epidemic modelling uses the framework of a
      \emph{compartmented model} of a disease}
    \begin{itemize}
    \item A number of compartments holding some fraction of the
      population
    \item (Can also think of this as the state of each individual)
    \item Rules on how these fractions change over time
    \end{itemize}
  \end{block}
\end{frame}

\begin{frame}
  \frametitle{Continuous SIR}
  
  \begin{columns}
    \begin{column}[t]{.45\textwidth}
      \begin{block}{The model}
          \begin{itemize}
          \item \emph{S}usceptible individuals can catch the infection
            from \emph{I}nfected individuals
          \item \ldots who then are \emph{R}emoved from the dynamics
            by recovery (or death) 
          \end{itemize}
        \end{block}
      \end{column}

      \begin{column}[t]{.45\textwidth}
        \begin{block}{Epidemic dynamics}
          \begin{itemize}
          \item Susceptibles infected per contact with probability
            $\beta$
          \item Infecteds removed with probability $\alpha$ 
          \end{itemize}
        \end{block}
      \end{column}
  \end{columns}

  \vspace{.5cm}
  $$
  \frac{dS}{dt} = -\beta SI \qquad
  \frac{dI}{dt} = \beta SI - \alpha I \qquad
  \frac{dR}{dt} = \alpha I
  $$
\end{frame}

\begin{frame}
  \frametitle{Solution}

  \centering
  \includegraphics[height=0.8\textheight,keepaspectratio]{sir-progress-dt}
\end{frame}

\begin{frame}
  \frametitle{A range of models with increasing complexity}

  \begin{block}{Can model different disease structures\cite{Hethcote-CompartmentedModels}}
    \begin{itemize}
    \item SIR -- simple infection offering complete immunity
      post-infection
    \item SIS -- infection confers no immunity
    \item SEIR -- exposed individuals are infectious before symptoms
    \item MSEIR -- initial immunity passed from mother to child
    \item SEIRS -- immunity wears off with time
    \item \ldots
    \end{itemize}
  \end{block}
\end{frame}


\subsection{Epidemics on networks}

\begin{frame}
  \frametitle{Epidemics on networks -- 1}

  \begin{columns}
    \begin{column}[c]{.5\textwidth}
      \begin{block}{Complete homogeneous mixing implies populations have no structure,
          which clearly isn't true}
        \begin{itemize}
        \item Not everyone meets everyone else
        \item Some are massively more (or less) connected than
          average
        \end{itemize}
      \end{block}      
    \end{column}
    
    \begin{column}[c]{.5\textwidth}
      \includegraphics[width=\textwidth,keepaspectratio]{irishcentral-superspreader}
    \end{column}
  \end{columns}
\end{frame}

\begin{frame}
  \frametitle{Epidemics on networks -- 2}
  \begin{block}{Use a network as the substrate for the epidemic\cite{NewmanEpidemicDisease}}
    \begin{itemize}
    \item Only nodes that are adjacent can interact
    \item Compartment = label on node
    \end{itemize}
  \end{block}

  \begin{block}{Leads to some more modelling decisions}
    \begin{itemize}
    \item Degree distribution: probability $p_k$ of node having degree
      $k$
    \item Often treat the mean degree $\langle k \rangle$ as ``representative''
    \item Topological fine structure: loops, modules, layers, \ldots
    \item Adaptive behaviour to change these features in response to
      the progression of the disease (\textit{e.g.}, quarantine)
    \end{itemize}
  \end{block}
\end{frame}

\begin{frame}
  \frametitle{How to do analysis}

  \begin{columns}
    \begin{column}[c]{.6\textwidth}
      \begin{block}{The ``gold standard'' is an analytic model with
          numerical validation}
        \begin{itemize}
        \item Find an analytic description for what happens under
          different infection parameters
        \item Run process on random networks with the given topology
        \item Lots of repetitions to squeeze out variance
        \item (Hopefully) sample points land on solutions of the equations\cite{PercolationHigherOrderClustering-PRE20}
        \end{itemize}
      \end{block}
    \end{column}

    \begin{column}[c]{.4\textwidth}
      \includegraphics[width=\textwidth,keepaspectratio]{gcc-cliques}
    \end{column}
  \end{columns}
\end{frame}

\begin{frame}
  \frametitle{How to do numerical validation}

  \begin{block}{A networked population}
    \begin{itemize}
    \item Set all to susceptible
    \item Change a fraction to infected, chosen at random
    \item Run the model rules until equilibrium
    \end{itemize}
  \end{block}
  
  \begin{block}{Discrete-event simulation\cite{Gil77}}
    \begin{itemize}
    \item After a time $\tau$ we  see an event $e$ (infection,
      removal, \ldots)
    \item Draw the next time and event from joint distribution
      $P(\tau, e)$
    \end{itemize}
  \end{block}
\end{frame}

\begin{frame}[fragile]
  \frametitle{Field lacks standard tooling}

  \begin{block}{So we built some :-)}
    \begin{itemize}
    \item \texttt{epydemic}: models, standard topologies, \ldots
    \item \texttt{epyc}: experiments, parallelism, dataset management
    \end{itemize}
  \end{block}

  \begin{lstlisting}
lab = epyc.ClusterLab(profile='hogun')

lab[epydemic.ERNetwork.N] = 10000
lab[epydemic.ERNetwork.KMEAN] = 40
        
lab[epydemic.SIR.P_INFECTED] = 0.001
lab[epydemic.SIR.P_REMOVE] = 0.002
lab[epydemic.SIR.P_INFECT] = numpy.linspace(0.00001, 0.0002, num=50)
        
m = epydemic.SIR()
g = epydemic.ERNetwork()
e = epydemic.StochasticDynamics(m, g)
        
lab.runExperiment(epyc.RepeatedExperiment(e, 100))
  \end{lstlisting}

  \hfill{\tiny \url{https://github.com/simoninireland}}
\end{frame}


% ---------- Experiments ----------

\section{Exploring diseases on networks}

\subsection{Changing the contact network}

\begin{frame}
  \frametitle{What do human contact networks look like?}

  \begin{block}{Networks with particular contact structure}
    \begin{itemize}
    \item Tend to have cycles (friends of friends)
    \item Highly variable numbers of contacts
    \item Often modular, having different local and global structures 
    \end{itemize}
  \end{block}

  \begin{block}{Construction}
    \begin{itemize}
    \item Theoretically, using a process to generate the structures
    \item Empirically, structured from survey data\cite{Polymod}
    \end{itemize}
  \end{block}
\end{frame}

\begin{frame}
  \frametitle{ER networks and the epidemic threshold}

  \begin{block}{Erd\H{o}s-R\'enyi (ER) networks}
    \begin{itemize}
    \item For $N$ nodes build the complete network $K_N$
    \item For each edge, retain (``occupy'') it with probability
      $p_{infect}$
    \item Leads to $p_k$ normally distributed around $\langle k
      \rangle = p_{infect}N$ 
    \end{itemize}
  \end{block}

  \centering
  \hfill\includegraphics[width=.35\textwidth,keepaspectratio]{thresholds-er-raw} \hfill
  \includegraphics[width=.35\textwidth,keepaspectratio]{thresholds-er-errorbars} \hfill~
\end{frame}

\begin{frame}
  \frametitle{Not all networks behave like this}

  \begin{block}{Too ``even'' to be a good model of human contacts}
    \begin{itemize}
    \item Powerlaw with cutoff, $p_k \propto k^{-\alpha} e^{K/\kappa}$

      \centering
      \hfill\includegraphics[width=.35\textwidth,keepaspectratio]{hcn-large-kappa} \hfill
      \includegraphics[width=.35\textwidth,keepaspectratio]{hcn-small-kappa} \hfill~
     
    \item Relatively insensitive to $p_{infect}$, but sensitive to
      $\alpha$ and $\kappa$ 
    \end{itemize}
  \end{block}

\end{frame}


\subsection{Immunity}

\begin{frame}
  \frametitle{Herd immunity}

  \begin{columns}
    \begin{column}[c]{.6\textwidth}
      \begin{block}{Sufficient immune/recovered individuals to stop an
          epidemic propagating}
        \begin{itemize}
        \item Infecteds never adjacent to enough susceptibles
        \item First epidemic changes the effective topology
        \item ``Effective'' $\langle k \rangle$ drops from 20 to 5.5
        \item (Interested to explore other topological changes)
        \end{itemize}    
      \end{block}
    \end{column}
    \begin{column}[c]{.4\textwidth}
      \includegraphics[width=\textwidth,keepaspectratio]{herd-progress}
      \\[.5cm]
      \includegraphics[width=\textwidth,keepaspectratio]{herd-finals}
    \end{column}
  \end{columns}
\end{frame}

\begin{frame}
  \frametitle{Why this is a bad idea}

  \begin{block}{Herd immunity has been seriously advocated as a
      strategy for \textsc{covid-19}\footnote{See the ``Great
        Barrington Declaration'', \url{https://gbdeclaration.org}}}
  \end{block}

  \begin{block}{Ignores some rather inconvenient facts}
    \begin{itemize}
    \item A 1\% death rate = 700K UK deaths, about one year's excess
    \item At a rate that would collapse health services
    \item Immunity may not be permanent -- which makes herd immunity
      behave differently (or not appear at all)
    \item Long \textsc{covid} not accounted for in the costs
    \end{itemize}
  \end{block}
\end{frame}

\begin{frame}
  \frametitle{Vaccination}

  \begin{block}{``Herd immunity without the bad bits''}
    \begin{itemize}
    \item Aim for the herd immunity threshold, generally about 60\% of
      the population
    \item \ldots without anyone actually being infected
    \end{itemize}
  \end{block}
  
  \begin{columns}
    \begin{column}[c]{.65\textwidth}
      \begin{block}{Epidemic proceeds at different rates depending on
          topology}
        \begin{itemize}
        \item ``Enough'' contacts stabilise the size of outbreak 
        \end{itemize}
      \end{block}

    \end{column}
    \begin{column}[c]{.35\textwidth}
       \includegraphics[width=\textwidth,keepaspectratio]{vaccination-sir-hcn}
    \end{column}
  \end{columns}
\end{frame}

\begin{frame}
  \frametitle{Vaccination strategies}

  \begin{columns}
    \begin{column}[c]{.7\textwidth}
       \begin{block}{Randomly vaccinate 60\% of the population}
        \begin{itemize}
        \item Massive reduction in epidemic size
        \item Only catching high-degree nodes at random
        \item Sensitive to missing people
        \end{itemize}
      \end{block}

      \begin{block}{If we target vaccination we can reduce the
          threshold needed to get the same effect}
        \begin{itemize}
        \item Target highest-degree 2\% of nodes
        \item Take out the super-spreaders
        \end{itemize}
      \end{block}
    \end{column}
    \begin{column}[c]{.3\textwidth}
      \includegraphics[width=\textwidth,keepaspectratio]{vaccination-60}
       \\[.5cm]
       \includegraphics[width=\textwidth,keepaspectratio]{vaccination-super}      
    \end{column}
  \end{columns}
\end{frame}


\subsection{Physical countermeasures}

\begin{frame}
  \frametitle{Physical distancing}

  \begin{block}{What does a physically-distanced network look like?}
    \begin{itemize}
    \item Good question
    \item (I have two SH projects looking at this at University level)
    \end{itemize}
  \end{block}

  \begin{block}{One possible model}
    \begin{itemize}
    \item Normally-distributed, fully connected family ``bubbles'' of mean size 4
    \item A couple of members with outside contacts
    \item Exponentially-distributed connections between the contacts in different bubbles
    \end{itemize}
  \end{block}
\end{frame}

\begin{frame}
  \frametitle{Lockdown changes propagation}

  \begin{columns}
    \begin{column}[c]{.7\textwidth}
      \begin{block}{Changes the epidemic threshold compared to an ER network}
        \begin{itemize}
        \item Needs a higher infectivity to take off
        \end{itemize}
      \end{block}

      \begin{block}{Slower take-off}
        \begin{itemize}
        \item Not like a powerlaw network
        \item Get bursts of infection if the infection gets into a bubble
        \end{itemize}
      \end{block}
    \end{column}
    \begin{column}[c]{.3\textwidth}
      \includegraphics[width=\textwidth,keepaspectratio]{distancing-no}       
      \\[.5cm]
      \includegraphics[width=\textwidth,keepaspectratio]{distancing-slower}       
    \end{column}
  \end{columns}
\end{frame}


% ---------- SEIR ----------

\section{Track and trace}

\subsection{SEIR infections}

\begin{frame}
  \frametitle{Asymptomatic transmission}

  \begin{block}{Because \textsc{covid-19} is essentially SEIR (or
      maybe SEIRS) it invites other countermeasures}
    \begin{itemize}
    \item Self-isolating on showing symptoms is ineffective
    \item Try to find the asymptomatic carriers
    \end{itemize}
  \end{block}

  \begin{block}{This is the basis for track-and-trace}
    \begin{itemize}
    \item Identify contacts of that person
    \item Quarantine them if they're infected -- means we catch
      infecting individuals before they knew to self-isolate
    \item Quarantine the symptomatic individual too
    \end{itemize}
  \end{block}
\end{frame}

\begin{frame}
  \frametitle{Track and trace in practice}

  \begin{block}{A large-scale procedure, unlike the local procedure of
      self-isolation when symptomatic}
    \begin{itemize}
    \item Requires organisation by some authority
    \item What can possibly go wrong?\ldots
    \end{itemize}
  \end{block}

  \begin{block}{Unlikely to be fully accurate even if done
      competently}
    \begin{itemize}
    \item Some proportion of people don't quarantine? ($p_{rewire}$)
    \item Only test some proportion of contacts? ($p_{detect}$)
    \end{itemize}
  \end{block}
\end{frame}

\begin{frame}
  \frametitle{The impact of detection rates}

  \begin{block}{Hold $p_{rewire}$ constant and vary $p_{detect}$}
  \end{block}

  \centering
  \includegraphics[width=.6\textwidth,keepaspectratio]{seir-waterfall}  
\end{frame}


\subsection{Wild speculations}

\begin{frame}
  \frametitle{Interpreting the plot}

  \begin{columns}
    \begin{column}[c]{.6\textwidth}
      \begin{block}{Very dependent on detection rate}
        \begin{itemize}
        \item Quarantine can still sometimes damp epidemic anyway
        \item High detection is very effective
        \end{itemize}
      \end{block}
      
      \begin{block}{Seems to be unstable in the mid range}
        \begin{itemize}
        \item All sizes of epidemic are possible
        \item No clean break
        \item Phase transition is ``smeared out'' rather than
          occurring cleanly
        \end{itemize}
      \end{block}
    \end{column}
    \begin{column}[c]{.4\textwidth}
       \includegraphics[width=\textwidth,keepaspectratio]{seir-waterfall}
    \end{column}
  \end{columns}
\end{frame}

\begin{frame}
  \frametitle{What's happening?}

  \begin{block}{This phenomenon only entered the literature last year\cite{Smeared-19}}  
  \end{block}

  \begin{block}{Possible explanations}
    \begin{itemize}
    \item It's an artefact of ER networks
    \item It's an artefact of not simulating long enough to crush
      the variance
    \item It's a function of network fine structure like clustering
    \item SEIR propagation is really driven by asymptomatic infection
    \end{itemize}
  \end{block}
\end{frame}

\begin{frame}
  \frametitle{Possible implications}

  \begin{block}{Designing track-and-trace}
    \begin{itemize}
    \item Very sensitive to the proportion of contacts tested
    \item Need to check at least 40\% to have any effect at all
    \end{itemize}
  \end{block}

  \begin{block}{Effectiveness}
    \begin{itemize}
    \item Even very effective tracing doesn't guarantee elimination
    \item Although it does reduce the peak significantly once above
      about 80\%
    \item Understanding the smearing might let us improve the test
      strategy
    \end{itemize}
  \end{block}
\end{frame}

\begin{frame}
  \frametitle{Four things to take away}

  \begin{enumerate}
  \item Epidemic spreading still isn't fully understood, even now --
    there's lots of exciting work still to do, mathematically and computationally
  \item Explore of possible public policy decisions 
  \item Need (in my copious free time) to understand what's going on
    with this possible smeared phase transition
  \item Especially interested in how small-scale topological
    structures affect network-based processes
  \end{enumerate}
\end{frame}


% ---------- Endmatter ----------

\begin{frame}
  \frametitle{References}

  \bibliographystyle{abbrvnat}
  {
    \tiny
    \bibliography{smeared}
  }
\end{frame}

\end{document}
